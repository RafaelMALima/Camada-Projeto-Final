
\documentclass{article}
\usepackage[utf8]{inputenc}
\usepackage[T1]{fontenc}
\usepackage{geometry}
\usepackage{amsmath}
\usepackage{amssymb}
\geometry{a4paper}
\title{The evolution of mobile data thecnologies, from 2G to 5G\\
	\large Tema da trabalho \\}
\author{André Melo, Rafael Lima e Rodrigo Anciães}
%\date{data}
\begin{document}

\maketitle
\begin{abstract}
Within the past 40 years, innovations in the field of mobile wireless communications have changed the way data is transmitted digitally to an extreme extent.   
\end{abstract}
\pagebreak

\section{2G and it's innovations}
2G's innovations over it's predecessors lay in it's use of digital signals instead of analogue. While 1G used frequency modulation at a band of band of 824-894MHz to encode information, 2G used TDMA and CDMA for it's modulation, having a bitrate of 64kbps, and a bandwith of 30-200KHz. Since 2G encodes it's signals digitally, cellphone technologies other than phone calls, such as the SMS.

\section{3G}

\section{4G}

\section{5G}

\end{document}
