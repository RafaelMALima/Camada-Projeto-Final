
\documentclass{article}
\usepackage[utf8]{inputenc}
\usepackage[T1]{fontenc}
\usepackage{geometry}
\usepackage{amsmath}
\usepackage{amssymb}
\geometry{a4paper}
\title{The evolution of mobile data thecnologies, from 2G to 5G\\
	\large Tema da trabalho \\}
\author{André Melo, Rafael Lima e Rodrigo Anciães}
%\date{data}
\begin{document}

\maketitle
\begin{abstract}
Mobile communications technologies utilize radio frequencies to be able to perform a range of different things, from voice calls to content drastically changed the way mobile data is transmitted.
\end{abstract}
\pagebreak

\section{2G and it's innovations over previous mobile data technologies}
2G's innovations over it's predecessors lay in it's use of digital signals instead of analogue. While 1G used frequency modulation at a band of band of 824-894MHz to encode information, 2G used mostly TDMA and CDMA for it's modulation schemes and had a bitrate of 64kbps, and a bandwith of 30-200KHz. While TDMA has only a single channel, that allocates timeslots for each user transmitting data, CDMA dealt with user signal division by giving each user an uniquely ID'd channel(citation needed). Since it's signals are digital, cellphone communication technologies other than phone calls became possible, such as the SMS. 2G's most widely used modulation scheme whowever, GSM (Global System for Mobile Communication), used a mix of TDMA and FDMA. It's total bandwith was divided into multiple frequency channels, and each channel was then divided using the TDMA scheme[1]

\section{3G's incremental approach to improving 2G}
While 2G's innovations over it's predecessors where massive, the process of innovation from 2G to 3G was much more incremental. 3G's purpose is to serve as an improved version of 2G, with faster data transfer speeds of up to 2Mbps, while having a frequency band of 15-20MHz, making it's use possible for applications such as web browsing and other more data heavy applications.

\section{4G}

\section{5G and it's future and Challenges}
5g and its future and challanges
Differently from some of its predecessors, 5G aims to be a massive step forward its predecessors focusing on causing a revolutionary impact in terms of data rates, latency, massive connectivity, network reliability, and energy efficiency, however, its advances require whole new structures and have its particular issues and trials. Due to the saturation of the already-in-use frequency bands and its consequent lack of bandwidth, 5G is targeting the frequency band in the mm-wave range from 24-100 GHz, creating, due to its approval,a great communication-only band[2]

\section{References}
[1]Javier Gozálvez Sempere"An overview of the GSM system"
[2]Mansoor Shafi, Andreas F. Molisch, Peter J Smith, Thomas Haustein, Peiying Zhu, Prasan De Silva, Fredrik
Tufvesson, Anass Benjebbour, Gerhard Wunder"5G: A Tutorial Overview of Standards, Trials,
Challenges, Deployment and Practice"

\end{document}