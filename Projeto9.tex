
\documentclass{article}
\usepackage[utf8]{inputenc}
\usepackage[T1]{fontenc}
\usepackage{geometry}
\usepackage{amsmath}
\usepackage{amssymb}
\geometry{a4paper}
\title{The evolution of mobile data thecnologies, from 2G to 5G\\
	\large Tema da trabalho \\}
\author{André Melo, Rafael Lima e Rodrigo Anciães}
%\date{data}
\begin{document}

\maketitle
\begin{abstract}
Within the past 40 years, innovations in the field of mobile wireless communications have changed the way data is transmitted digitally to an extreme extent.   
\end{abstract}
\pagebreak

\section{2G and it's innovations}
2G's innovations over it's predecessors lay in it's use of digital signals instead of analogue. While 1G used frequency modulation at a band of band of 824-894MHz to encode information, 2G used mostly TDMA and CDMA for it's modulation schemes and had a bitrate of 64kbps, and a bandwith of 30-200KHz. While TDMA has only a single channel, that allocates timeslots for each user transmitting data, CDMA dealt with user signal division by giving each user an uniquely ID'd channel(citation needed). Since it's signals are digital, cellphone communication technologies other than phone calls became possible, such as the SMS. 2G's most widely used modulation scheme whowever, GSM, used a mix of TDMA and FDMA. It's total bandwith was divided into multiple frequency channels, and each channel was then divided using the TDMA scheme[1]

\section{3G}

\section{4G}

\section{5G}

\section{References}
[1]Javier Gozálvez Sempere"An overview of the GSM system"

\end{document}